\documentclass[11pt,a4paper]{article}

\usepackage[left=2cm,text={17cm, 24cm},top=3cm]{geometry}
\usepackage{times}
\usepackage[utf8]{inputenc}
\usepackage[slovak]{babel}
\usepackage{xurl}
\usepackage[breaklinks]{hyperref}
\PassOptionsToPackage{hyphens}{url}\usepackage{hyperref}
\urlstyle{same}

\begin{document}
\begin{titlepage}
\begin{center}
    \Huge\textsc{Vysoké učení technické v Brně}\\
    \huge\textsc{Fakulta informačních technologií}\\
    \vspace{\stretch{0.38}}
    \LARGE{Typografie a publikování\,--\,3. projekt\\}
    \Huge{Typografia a citácie\\}
    \vspace{\stretch{0.62}}
\end{center}
{\Large \today \hfill Jakub Vaňo(xvanoj00)}
\end{titlepage}

\section*{Typografia v marketingu}


Dobrá typografia je ako chlieb, pripravená na obdivovanie, hodnotenie a rozobranie skôr, ako sa skonzumuje~\cite{book-bringhurst}


Všeobecne typografia je umenie alebo schopnosť vytvoriť komunikáciu pomocou písanho slova. Solomon ju
definuje ako umenie mechanicky vytvorených písmen, čísel, symbolov a tvarov pomocou porozumenia
základných princípov, elementov a atribútov dizajnu. Následne Lupton ju popisuje ako návrh foriem písma a ich usporiadanie v~priestore~\cite{el-thangaraj}.


V postmodernom reklamnom dizajne sa text a obraz spájajú spolu v spletitej syntaxi, ktorá spája typografiu, fotografiu, grafy, grafiku a text na vytvorenie správy.~\cite{mag-bartal}


Výskum ukázal, že typografia ovplyvňuje schopnosť ľudí spracovávať informácie z reklamy. McCarthy a~Mothersbaugh skúmali čitateľnosť typografie, ktorá je operacionalizovaná vzhľadom na veľkosť, výšku a štýl písma (serif vs. sans-serif).  Berú do úvahy schopnosť čitateľa (pomalý vs. rýchly) a ukazujú, že rýchli čitatelia čítajú viac slov, keď je text prezentovaný serif typom písma a vysokou výškou x (výška malého x v konkrétnej typografii) v porovnaní s textom prezentovaným sans-serif typom písma~\cite{el-amar}.


Daniel Starch vo svojej štúdii porovnával dve reklamy na cigarety Lucky Strike. V
jednom prípade bol titulok vytlačený horizontálne a v druhom na šikmo. Výsledkom sa
zistilo, že čitatelia pochopili text v oboch prípadoch, no ľahšie čitateľné boli pre nich
horizontálne znaky~\cite{thesis-suppenova}


Na získanie zákazníkovej pozornosti je nevyhnutné mať jedno kontaknté miesto. Ústredným bodom je najčastejšie obrázok, pričom nadpis a podnadpis pôsobia ako sekundárne prvky.~\cite{book-white}.


Trik, ktorý je často účinný v exteriéry, je neónový nápis. V noci, keď je tma, akýkoľvek jasný kúsok svetla priťahuje oči, ale keď sa neónové nápisy používajú masovo, ich účinnosť začína zlyhávať a sú len tie s najlepším umiestnením ostávajú efektívne~\cite{serial-kurt}


V marketingovom dizajne boli zaznamenané dva trendy. Jedným z nich je použitie línií obrazu s nepravidelnými okrajmi, ktoré sa obopínajú okolo obrázku pričom spájajú tieto dva elementy dokopy. Druhá technika spočíva v tvarovaní reklamy do siluety obrázka. Tvar písma tak nadobúda svoj vlastný význam. Oba však spájajú dizajn na úkor čitateľnosti správy~\cite{serial-moriarty}


Ďalším štýlom zahŕňajúcim expresívnu typografiu je dekonštruktívna typografia. Je to štýl inšpirovaný modernou lingvistickou teóriou, ktorá patrí k širšiemu štýlu známemu ako postštrukturalizmus. Texty sú členené a vizualizované striedajúc kompozíciu a veľkosť písma~\cite{thesis-donev}.


V oblasti turizmu sa napríklad odporúča integrovať prvky umenia danej oblasti do turistickej reklamy, ktorá tak zlepšuje vizuálny obraz reklamy a podporuje turistickú interakciu s reklamou.~\cite{el-sedek}



\newpage
\bibliography{proj4}
\bibliographystyle{czechiso}

\end{document}
